\documentclass[12pt]{biom}

% Page setup
%\usepackage[margin=1in]{geometry}
%\usepackage{setspace}
%\doublespacing % ensures double spacing
%\setlength{\textheight}{9in} % ~25 lines per page with 12pt double spaced
%\setlength{\parskip}{0pt}
%\setlength{\parindent}{0.5in}

% Font
\renewcommand{\rmdefault}{ptm} % Times font (Biometrics-like)

% References (Biometrics style uses Chicago-like author-year)



 
\RequirePackage[colorlinks,citecolor=blue,urlcolor=blue]{hyperref}
\RequirePackage{graphicx}


%custom packages
\usepackage{algorithm}
\usepackage[noend]{algpseudocode}
\usepackage{hyperref}
\usepackage{enumerate}
\usepackage{xcolor}
\usepackage{bbm}
% \usepackage{url} % not crucial - just used below for the URL 
% \usepackage{bibunits}
\usepackage{subcaption}
% \usepackage{adjustbox}
\usepackage{verbatim} % to comment out

%\theoremstyle{plain}
%\newtheorem{axiom}{Axiom}
%\newtheorem{claim}[axiom]{Claim}
%\newtheorem{theorem}{Theorem}[section]
%\newtheorem{lemma}[theorem]{Lemma}
%\newtheorem{proposition}[theorem]{Proposition}
%\newtheorem{corollary}[theorem]{Corollary}%custom
%%%%%%%%%%%%%%%%%%%%%%%%%%%%%%%%%%%%%%%%%%%%%%
%%                                          %%
%% For Assumption, Definition, Example,     %%
%% Notation, Property, Remark, Fact         %%
%% use \theoremstyle{definition}            %%
%%                                          %%
%%%%%%%%%%%%%%%%%%%%%%%%%%%%%%%%%%%%%%%%%%%%%%
%\theoremstyle{definition}
%\newtheorem{definition}[theorem]{Definition}
%\newtheorem*{example}{Example}
%\newtheorem*{fact}{Fact}
%\newtheorem{remark}{Remark}



% Counters for figures/tables
%\usepackage{etoolbox}
%\newcounter{totalfigtab}
%\pretocmd{\begin{figure}}{\stepcounter{totalfigtab}}{}{}
%	\pretocmd{\begin{table}}{\stepcounter{totalfigtab}}{}{}
		
		% Title page


		\begin{document}
			

			

			

		
			

Biomedical studies often collect multi-modal data, as in the Emory University renal study \citep{changBayesianLatentClass2020, jangPrincipalComponentAnalysis2021}, which records multiple renogram curves (functional data) along with scalar side information for each kidney.
Such distinct yet related physiological signals can be represented as functional- and scalar-valued i.i.d. covariates in a linear regression model:
\begin{equation}	\label{eq:hybrid_regression_model}
	Y_i
	= 
	\boldsymbol{\beta}^\top \mathbf{Z}_i
	+ 
	\sum 
	\limits_{k=1}^K 
	\int  \beta_k(t) X_{ik}(t) dt + \epsilon_i,
	\quad
	i = 1, \ldots, n,
	%= \langle \boldsymbol{\alpha}, \mathbf{Z} \rangle +  \langle \beta, X \rangle_{\mathcal{F}}  + \epsilon
\end{equation}
where $Y_i$ is a scalar response, 
$X_{i1}(t), \ldots, X_{iK}(t) \in \mathbb{L}^2[0,1]$ are functional predictors, 
$\mathbf{Z}_i = (Z_{i1}, \ldots, Z_{ip})^\top$ is a Euclidean vector covariate,
and $\epsilon_i$ denotes observational noise.
%In other words, this is a scalar-on-hybrid regression where the hybrid covariates belong to $\bigl( \mathbb{L}^2[0,1] \bigr)^{  K} \times \mathbb{R}^p$.
For notational convenience, we assume throughout that the responses and predictors are centered, so the intercept term can be omitted.

In addition to the ill-posedness of the infinite-dimensional slope functions $\beta_k(t)$ and the high dimensionality of the predictors ($K + p$), a further challenge comes from the strong correlations between functional and scalar predictors. This paper addresses all three issues in a unified manner by introducing a hybrid partial least squares (PLS) regression framework defined on a novel hybrid Hilbert space.
	
	\subsection{Previous works and limitations}
	To address the ill-posedness stemming from the infinite-dimensionality of the functional components, common remedies include basis expansion,
	using power-series  \citep{goldsmithPenalizedFunctionalRegression2011},
	B-splines \citep{cardotSplineEstimatorsFunctional2003, caiPredictionFunctionalLinear2006}, or 
	wavelets \citep{zhaoWaveletbasedLASSOFunctional2012},
	and structure-aware regularization, such as roughness penalties  
	
	For the high-dimensionality, a major solution is using derived inputs.
	A straightforward approach in this direction is principal component analysis (PCA) regression, which applies PCA separately to the functional- (by, for example, \citealt{happ_multivariate_2018}) and scalar predictors, and then performs a classical multivariate regression on the combined scores.
	PCA regression is well-studied for linear regression with functional predictors alone \citep{hallMethodologyConvergenceRates2007, reissFunctionalPrincipalComponent2007, febrero-bandeFunctionalPrincipalComponent2017}.
	However, since the PCA-derived inputs are not informed by the response variable,   they are not guaranteed to capture the core regression relationship with a small number of derived inputs.
	In terms of predictive power, partial least squares (PLS) regression is a powerful alternative. It iteratively constructs a set of orthogonal latent components from the predictors that have the maximal covariance with the response variable, and use the resulting scores for regression.
	%
	PLS for linear regression in the context of functional predictor alone was introduced by \citet{predaPLSRegressionStochastic2005} for the case of a single predictor. Motivated by regression on chemometric spectra, \citet{reissFunctionalPrincipalComponent2007}, \citet{aguileraUsingBasisExpansions2010}, and \citet{aguileraPenalizedVersionsFunctional2016} extended the framework by incorporating basis approximations and roughness penalties to promote smoothness. \citet{delaigleMethodologyTheoryPartial2012} provided the first thorough theoretical analysis, while \citet{saricamPartialLeastsquaresEstimation2022} proposed a computationally efficient procedure based on Golub-Kahan bidiagonalization. For a comprehensive review, see \citet{febrero-bandeFunctionalPrincipalComponent2017}. More recently, \citet{beyaztasRobustFunctionalPartial2022} extended the framework to accommodate multiple functional predictors.
	
	However,  these existing approaches overlook the potentially strong correlations between functional and scalar components, which can lead to multicollinearity and suboptimal predictive performance. Multimodal correlations have mostly been addressed within an unsupervised learning framework. For instance, \citet{kolar_graph_2014} studied the estimation of joint undirected graphical models for  functional and vector data, while \citet{gengJointNonparametricPrecision2020} considered joint precision matrix estimation for brain measurements and confounding scalar covariate. \citet{jangPrincipalComponentAnalysis2021} proposed a joint PCA method that accounts for correlations between functional and scalar data. However, the resulting components are still not informed by the response and may fail to capture  correlation with the outcome.
	
	
	\subsection{Our contributions}
	To address the gaps mentioned above, we unify the remedies proposed for these three issues: basis expansion, partial least squares, and accounting for correlations between the functional and scalar components.
	We propose a hybrid PLS regression framework that integrates functional and vector predictors in a principled and coherent manner. To extract predictive structure from these jointly observed and potentially correlated data types, we define a Hilbert space that treats the tuple of functional and vector components as a single hybrid object, equipped with a suitable inner product. The hybrid PLS direction is then obtained by iteratively maximizing the empirical covariance with the response, subject to a unit-norm constraint in this Hilbert space. Our framework is readily applicable to dense and irregular functional data and supports regularization techniques to prevent overfitting and reduce variance. We also provide the mathematical properties that justify our algorithm.
	
	
	\section{Background on partial least squares and its extension to hybrid predictors}  

	For intuition, let us return to the high-dimensional Euclidean predictor setting $
	Y_i =  \boldsymbol{\beta}^\top \mathbf{Z}_i + \epsilon_i$.
	A common way to address ill-posedness and correlation is to approximate the high-dimensional vector $\mathbf{Z}_i$ using a low-dimensional vector $
	\bigr(
	\widehat{\rho}_1^{[1]}, \ldots,   \widehat{\rho}_1^{[L]}  
	\bigr)^\top \in \mathbb{R}^L.
	$ To retain the regression relationship, the $l$-th PLS direction $\hat{\boldsymbol{\xi}}_l$ solves:
	\begin{equation}\label{scalar_PLS_sample}
		\max_{ \mathbf{ h } } 
		\widehat{\operatorname{Cov}}^2
		\bigl(
		\{
		\langle 
		\mathbf{ h }, \mathbf{Z}_i
		\rangle
		, Y_i
		\}_{i=1}^n
		\bigr)~\text{s.t.}~\| \mathbf{ h } \|_2 = 1,~\mathbf{ h }^\top  \widehat{\operatorname{Cov}}^2
		\bigl(
		\{
		\mathbf{Z}_i
		\}_{i=1}^n
		\bigr) \hat{\boldsymbol{\xi}}_j = 0, \quad j = 1, \ldots, l-1,
	\end{equation}
	where the two $\widehat{\operatorname{Cov}}^2$ denote sample cross-covariance and sample covariance, respectively. A standard algorithm for solving this optimization problem, called nonlinear iterative partial least squares (NIPALS) is presented in Algorithm \ref{alg:scalar_pls}.
	\begin{algorithm} 
		\caption{Scalar partial least squares regression}\label{alg:scalar_pls}
		\begin{algorithmic}[1]
			\State Standardize each $\mathbf{Z}_1, \ldots, \mathbf{Z}_n$ so that each feature have mean zero and variance one. Standardize $Y_1, \ldots, Y_n$.
			\For{$l = 1, 2, \ldots, L$}
			\\
			\textbf{PLS direction and score estimation:}
			\State $\widehat{\boldsymbol{\xi}}^{[l]} \leftarrow
			\arg \max_{  \boldsymbol{\alpha}   } 
			\widehat{\operatorname{Cov}}^2
			\bigl(
			\{
			\langle 
			\boldsymbol{\alpha}, \mathbf{Z}^{[l]}_i
			\rangle
			, Y^{[l]}_i
			\}_{i=1}^n
			\bigr)~\text{s.t.}~\| \boldsymbol{\alpha} \|_2 = 1
			$
			\Comment{PLS direction}\label{alg:step:scalar_pls_direction}
			\State 
			$
			\widehat{\rho}_i^{[l]} \leftarrow \langle \hat{\boldsymbol{\xi}}^{[l]},   \mathbf{Z}^{[l]}_i \rangle, 
			~i=1,  \ldots, n
			$ \Comment{PLS score}
			\\
			\textbf{Residualization:}
			\State $\nu^{[l]} 
			\leftarrow
			\frac{
				\sum_{i=1}^n Y_i^{[l]} \widehat{\rho}_{i}^{[l]}}{
				\sum_{i=1}^n \widehat{\rho}_{i}^{[l]2}
			}$ \Comment{Least squares estimate} 
			%
			\State $ Y_i^{[l+1]} \leftarrow Y_i^{[l]} - \nu^{[l]}\widehat{\rho}_i^{[l]}~i=1,  \ldots, n$
			%
			\State $ \widehat{\boldsymbol{\delta}}^{[l]}  \leftarrow \frac{1}{\sum_{i=1}^n \widehat{\rho}_{i}^{[l]2}}\sum_{i=1}^n  \widehat{\rho}_{i}^{[l]}\mathbf{Z}_i^{[l]} $ \Comment{Least squares estimate} 
			\label{alg:step:scalar_pls_residualization}
			\State $\mathbf{Z}_i^{[l+1]}  \leftarrow \mathbf{Z}_i^{[l]} -   \widehat{\rho}_{i}^{[l]}  \widehat{\boldsymbol{\delta}}^{[l]} $
			\EndFor
			\textbf{Regression coefficient estimation:}
			\State
			$\widehat{\boldsymbol{\iota}}^{[1]} \leftarrow \widehat{\boldsymbol{\xi}}^{[1]}$
			\For{$l =  2, \ldots, L$}
			\State $\widehat{\boldsymbol{\iota}}^{[l]} \leftarrow \widehat{ \boldsymbol{\xi} }^{[l]} - \sum_{u=1}^{l-1} \langle \widehat{\boldsymbol{\delta}}^{[u]}, \widehat{\boldsymbol{\xi}}^{[l]} \rangle 
			\widehat{\boldsymbol{\iota}}^{[u]}$  
			\EndFor
			$	\widehat{\boldsymbol{\beta}} \leftarrow \sum \limits_{l=1}^L\widehat{\nu}^{[l]} \widehat{\boldsymbol{\iota}}^{[l]}$ \Comment{regression coefficient estimate}
			\State \textbf{Output:} the  regression coefficient estimate
		\end{algorithmic}
	\end{algorithm}
	
	To extend Algorithm \ref{alg:scalar_pls} to the hybrid regression model \eqref{eq:hybrid_regression_model}, we must compute hybrid objects in both the PLS direction estimation and predictor residualization steps (lines \ref{alg:step:scalar_pls_direction} and \ref{alg:step:scalar_pls_residualization}). These computations cannot be separated into functional and scalar parts, as their correlation must be addressed.  
	A naive approach is to add loops over the observed evaluation points $t$ to lines \ref{alg:step:scalar_pls_direction} and \ref{alg:step:scalar_pls_residualization}, treating  
	\[
	\bigl( X_{i1}(t), \ldots, X_{iK}(t), Z_{i1}, \ldots, Z_{ip} \bigr) \in \mathbb{R}^{K+P}
	\]  
	as a Euclidean predictor, applying steps \ref{alg:step:scalar_pls_direction} and \ref{alg:step:scalar_pls_residualization}, and aggregating the results. This pointwise method, however, is computationally prohibitive for densely observed functional predictors, infeasible for irregular data, and prone to variability across the domain, resulting in overfitting and unstable predictions.  
	The main challenge of extension to the hybrid setting lies in the following points. First, 
	independent variables consist of multiple highly structured images and scalar predictors. 
	Second, our sample size is small compared to the dimension and number of functional and scalar predictors.
	Third,  existing partial least squares (PLS) methods can only accommodate (i) univariate or multivariate functional predictors without any scalar predictors \citep{predaPLSRegressionStochastic2005, delaigleMethodologyTheoryPartial2012, febrero-bandeFunctionalPrincipalComponent2017, Beyaztas2020}; or (ii) a univariate functional predictor with other scalar predictors \citep{Wang2018}.
	The next section proposes a new extension framework that addresses these issues.
	
	
	
	
	
	
	
	\section{Proposed PLS Algorithm} \label{section:main:our_algorithm}
	Our new framework builds on the fact that all computations in Algorithm \ref{alg:scalar_pls} are arithmetic operations in an inner product space. We therefore formally define the Hilbert space of hybrid random predictors along with its corresponding complete inner product, and leverage them to extend Algorithm \ref{alg:scalar_pls}.
	\begin{definition}[Hybrid space]\label{def:hilbert_space}
		Let $\mathbb{H}$ be a product space defined as the Cartesian product of $K$ copies of the space of square-integrable functions on $[0,1]$ and the $p$-dimensional Euclidean space:
		$$
		\mathbb{H} := (\mathbb{L}^2[0,1])^K \times \mathbb{R}^p.
		$$
		An element $h \in \mathbb{H}$ is an ordered tuple $h = (f_1, \dots, f_K, \mathbf{u})$, where $f_k \in \mathbb{L}^2[0,1]$ for $k=1, \dots, K$ and $\mathbf{u} \in \mathbb{R}^p$. This space is equipped with element-wise vector addition and scalar multiplication.
		We define an inner product on $\mathbb{H}$ for any two elements $h_1 = (f_1, \dots, f_K, \mathbf{u})$ and $h_2 = (g_1, \dots, g_K, \mathbf{v})$ as:
		\begin{equation}\label{eq: hybrid inner product}
			\langle h_1, h_2 \rangle_{\mathbb{H}} := \sum_{k=1}^K \int_0^1 f_k(t) g_k(t) \, dt + \omega \mathbf{u}^\top \mathbf{v},
		\end{equation}
		where $\omega$ is a positive constant ($\omega > 0$).
		The inner product induces a norm $\Vert \cdot \Vert_{\mathbb{H}}$ on the space, defined as $\Vert h \Vert_{\mathbb{H}} := \langle h, h \rangle_{\mathbb{H}}^{1/2}$, and a corresponding metric $d(h_1, h_2) = \Vert h_1 - h_2 \Vert_{\mathbb{H}}$.
	\end{definition}
	In \eqref{eq: hybrid inner product}, 
	$\omega$ is a positive weight that needs to be pre-specified or estimated. It is mainly used to take into account heterogeneity between functional and scalar parts in terms of measurement scale and/or amount of variation (see Section \ref{subsec: Data Preprocessing}).
	Without loss of generality and for the clarity of illustration, all the following theoretical results will be derived for $\omega=1$. The results remain valid for any positive weights. 
	\begin{lemma}[Hybrid Hilbert space]\label{lemma:hybrid_hilbert_space}
		The hybrid space $\mathbb{H}$ is a separable Hilbert space.
	\end{lemma} 
	Proof of Lemma \ref{lemma:hybrid_hilbert_space} is provided in Appendix \ref{section:proof:lemma:hybrid_hilbert_space}.
	
	\begin{definition}[Hybrid Predictor]\label{def:hybrid_predictor}
		For the Hilbert space  $\mathbb{H}$ 
		defined in Definition \ref{def:hilbert_space},
		The Borel $\sigma$-field on $\mathbb{H}$, denoted $\mathfrak{B}(\mathbb{H})$, is the smallest $\sigma$-field containing the class $\mathfrak{M}$ of all sets of the form
		$
		\{q \in \mathbb{H} \mid \langle q, h \rangle \in O\},
		$
		for any $h \in \mathbb{H}$ and any open subset $O \subseteq \mathbb{R}$ (details can be found in Theorem 7.1.1 of \citealt{hsingTheoreticalFoundationsFunctional2015}).
		A hybrid predictor $ W_i = (X_{i1}(t), \ldots, X_{iK}(t), \mathbf{Z}_i)$ is a measurable mapping from a probability space $(\Omega, \mathfrak{F}, P)$ into   $
		\bigl(
		\mathbb{H}, \mathfrak{B}(\mathbb{H})
		\bigr)
		$. 
	\end{definition}
	Then the joint regression model \eqref{eq:hybrid_regression_model}  can be concisely written as
	\begin{equation}		\label{eq: Hybrid functional model}
		Y_i = \langle \beta, W_i \rangle_{\mathbb{H}} + \epsilon_i,~\text{where}~\beta := \bigl( \beta_1(t), \ldots, \beta_K(t), \boldsymbol{\beta} \bigr)\in \mathbb{H}.
	\end{equation}
	\begin{remark}
		Our method is applicable to settings where each functional predictor belongs to a different Hilbert space, possibly defined over a distinct compact domain in $\mathbb{R}^d$, for arbitrary $d$, and observed at different time points. However, for notational simplicity, our discussion assumes a common Hilbert space over domain $[0,1]$ for all functional predictors.
	\end{remark}  
	
	Our approach provides an efficient and robust means of producing PLS components and scores in the presence of multiple dense and/or irregular functional predictors and scalar predictors. It also incorporates a regularization scheme that enables the algorithm to borrow strength and exploit structural relationships within and between the functions to avoid overfitting of the PLS components and to improve the generalizability and interpretability of the predictive model.
	Each iteration of our approach consists of two subroutines: regularized estimation of smoothed PLS components and orthogonalization, detailed in Sections~\ref{section:sub:compute_PLS_component} and~\ref{section:sub:residualization}, respectively. After a suitable number of iterations, the hybrid regression coefficient is estimated, as described in Section~\ref{section:sub:regression_coeff}. For notational simplicity, we omit the iteration index $l$ in the following discussion, with the understanding that the subroutines apply to any iteration. The complete algorithm is summarized in Algorithm~\ref{alg:hybrid_pls}.
	
	\subsection{Preliminary step 1: finite-basis approximation}\label{sec: finite basis approximation}	
	Let ${b_m(t)}$ be a twice-differentiable basis of $\mathbb{L}^2([0,1])$ whose second derivatives are also linearly independent, for example, cubic B-splines, the Fourier basis, or an orthonormal polynomial basis of degree greater than three. Using this basis, 
	the $j$th functional predictor, 
	regression coefficient, PLS component direction, and orthogonalization regression coefficient (with iteration indices suppressed) are represented as follows:
	\begin{equation*}
		X_{ij}(t) = \sum_{m=1}^\infty \theta_{ijm} \, b_m(t),~
		\beta_j(t) = \sum_{m=1}^\infty \eta_{jm} \, b_m(t),~
		%%%%%%%%%%%%%%%%%%%%%%
		\xi_j(t)
		=
		\sum_{m=1}^\infty
		\gamma_{jm}
		b_m(t),
		~
		\delta_j(t)
		=
		\sum_{m=1}^\infty
		\pi_{jm}
		b_m(t)		
	\end{equation*}	
	In practice, the full set of coefficients can not be obtained  with finite sample size, as functional data are measured on a finite grid. Thus, we truncate the expansion at $M$ terms.
	We choose a moderately large $M$ (e.g., 15 or 20) to capture functional variation without fine-tuning, as smoothness is handled via penalization (see Section~\ref{sec: Regularization of PLS Components}).
	The truncated expansions of the predictor and coefficient are denoted as
	\begin{equation*}
		\widetilde{X}_{ij}(t) := \sum_{m=1}^M \theta_{ijm} \, b_m(t), \quad
		\tilde{\beta}_j(t) := \sum_{m=1}^M \eta_{jm} \, b_m(t),
	\end{equation*}
	and our suggest method restricts each PLS component direction and orthogonalization regression coefficient to admit the following expansion:
	\begin{equation}\label{def:truncated_zeta_and_delta}
		\xi_j(t) = \sum_{m=1}^M \gamma_{jm} \, b_m(t),
		\quad
		\delta_j(t)
		=
		\sum_{m=1}^M
		\pi_{jm}
		b_m(t)		
	\end{equation}
	This implies that all computations in this paper are carried out entirely within the subspace
	\begin{equation}\label{def:truncated_hilbert_space}
		\widetilde{ \mathbb{H} } := \operatorname{span}\bigl(b_1(t), \ldots, b_M(t)\bigr)^K \times \mathbb{R}^p \subset \mathbb{H}.
	\end{equation}
	The $i$th hybrid predictor, projected on $\widetilde{ \mathbb{H} }$, is represented by the tuple
	\begin{equation*}
		\widetilde{W}_i := 
		(
		\widetilde{X}_{i1}, \ldots, \widetilde{X}_{iK}, \boldsymbol{Z}_i
		).
	\end{equation*}
	Let $\boldsymbol{\theta}_{ij}$, $\boldsymbol{\eta}_j$,  $\boldsymbol{\gamma}_j$, and
	$\boldsymbol{\pi}_j$
	denote the $M$-dimensional vectors of coefficients:
	\begin{equation}\label{def:coef_vector_form}
		\boldsymbol{\theta}_{ij} := (\theta_{ij1}, \ldots, \theta_{ijM})^\top,
		~
		\boldsymbol{\eta}_{j} := (\eta_{j1}, \ldots, \eta_{jM})^\top ,~
		\boldsymbol{\gamma}_{j} := (\gamma_{j1}, \ldots, \gamma_{jM})^\top,~
		\boldsymbol{\pi}_{j} := (\pi_{j1}, \ldots, \pi_{jM})^\top
	\end{equation}
	For the predictors, we stack the coefficient vectors across observations into the matrix
	\begin{equation}\label{def:theta}
		\Theta_j := (\boldsymbol{\theta}_{1j}, \ldots, \boldsymbol{\theta}_{nj})^\top  \in \mathbb{R}^{n \times M},
	\end{equation}
	and construct the full design matrix 
	\begin{equation}\label{def:full_theta}
		\Theta := (\Theta_1, \ldots, \Theta_K, \mathbf{Z}) \in \mathbb{R}^{n \times (MK + p)}.
	\end{equation}
	Let us denote the response vector as $\mathbf{y} := (y_1, \ldots, y_n)^\top$.
	Let $B, B^{\prime\prime} \in \mathbb{R}^{M \times M}$ denote the Gram matrices of the basis functions and their second derivatives, with entries
	\begin{equation}\label{def:B_matrix}
		B_{m, m'} := \int_0^1 b_m(t)\, b_{m'}(t)\, dt, \quad
		B^{\prime\prime}_{m, m'} := \int_0^1 b_m''(t)\, b_{m'}''(t)\, dt,
	\end{equation}
	for $m, m' = 1, \ldots, M$. We then define the block-diagonal matrices
	\begin{equation}\label{def:gram_block}
		\mathbb{B} := \operatorname{blkdiag}(B, \ldots, B, I_p), \quad
		\mathbb{B}^{\prime\prime} := \operatorname{blkdiag}(B^{\prime\prime}, \ldots, B^{\prime\prime}, I_p),
	\end{equation}
	
	Then the full data for the hybrid PLS problem at the $l$-th iteration can be represented by the tuple
	\begin{equation}\label{def:problem_instance}
		(\mathbb{B}, \mathbb{B}^{\prime\prime}, \Theta, \mathbf{y}) \in 
		\mathbb{R}^{(MK+p) \times (MK+p)} \times 
		\mathbb{R}^{(MK+p) \times (MK+p)} \times 
		\mathbb{R}^{n \times (MK+p)} \times 
		\mathbb{R}^n,
	\end{equation}
	with the index $l$ omitted for brevity.
	
	
	\begin{remark}
		While different bases could be used for each functional predictor, we adopt a common basis for simplicity. The definitions of $\mathbb{B}$ and $\mathbb{B}^{\prime\prime}$ remain general enough to accommodate distinct bases if needed.
	\end{remark}
	
	%Additionally, it serves as a regularization mechanism, borrowing strength across components and helping to prevent overfitting. 	
	
	
	
	\subsection{Preliminary step 2 : data preprocessing} \label{subsec: Data Preprocessing}
	Functional and scalar elements of the hybrid predictors often have incompatible units and/or exhibit different amounts of variation. This can be problematic for our PLS framework which is not scale invariant as: i) each predictor has different chance of contributing to the predictor/response structure; and ii) a predictor with high correlation to $Y$ but relatively low variance may be overlooked. 
	
	To obtain PLS components that have a meaningful interpretation, we standardize the predictor data via the following steps. The first step is to account for discrepancies \textit{within} respective functional and scalar parts, if needed. If the functional parts $\widetilde{X}_{i1}(t), \ldots, \widetilde{X}_{iK}(t)$ are measured in different units or have quite different domains, one can standardize them to have mean zero and integrated variance of one. If multivariate scalar predictors $\mathbf{Z}_i=(Z_{i1}, \ldots, Z_{ip})^\top$ exhibit different amounts of variation, one can standardize them to have mean zero and unit variance. The second step is to eliminate the discrepancies \textit{between} functional and scalar parts. To accomplish this aim, we choose an appropriate weight $\omega$ in the hybrid inner product \eqref{eq: hybrid inner product} that ensures functional and vector parts have comparable variance. A sensible data-driven approach to choosing an appropriate weight is to set
	\begin{equation*}
		\omega = \frac{\sum_{i=1}^n \sum_{k=1}^K \Vert \widetilde{X}_i \Vert^2_{\mathbb{L}^2[0,1]}}{\sum_{i=1}^n \Vert \mathbf{Z}_i \Vert_2^2},
		\label{eq: weight}
	\end{equation*}
	In practice, 
	instead of using inner product weighted by $\omega^{1/2}$, one can implement 
	this weighting scheme   by formulating the hybrid object as $\widetilde{W}_i = \bigl(
	\widetilde{X}_{i1}(t), \ldots, \widetilde{X}_{iK}(t),  \omega^{1/2} \mathbf{Z}_i
	\bigr)$, whose vector part has been scaled by a factor of $\omega^{1/2}$.
	
	\subsection{Iterative steps}\label{section:sub:iterative}
	The iterative process presented here yields an orthonormal hybrid basis that effectively captures the predictor-response relationships. It proceeds through two intermediate steps: the estimation of the PLS component direction (Section \ref{section:sub:compute_PLS_component}) and residualization (Section \ref{section:sub:residualization}). 
	The proofs are deferred to Appendix \ref{section:proof:section:sub:iterative}.
	The properties of the resulting estimates are introduced in Section \ref{section:main:properties_of_our_method}.
	\subsubsection{Iterative step 1: regularized estimation of  PLS component direction}  \label{section:sub:compute_PLS_component} 
	We begin by formally introducing the core optimization problem pertinent to the PLS direction estimation, which is formulated as a generalized Rayleigh quotient (Proposition \ref{proposition:eigen_noregul}). Building upon this foundational concept, we present our regularized PLS component direction estimation step that promotes smoothness (Proposition \ref{proposition:eigen_regul}). Furthermore,  we detail an efficient computational scheme (Proposition \ref{proposition:linear_regul}).
	
	\medskip
	\noindent
	\textit{Core optimization problem.}\label{sec: pls component computation intermediate}~
	We   present the core optimization problem that directly estimates the hybrid PLS component direction. It fully leverages the continuous nature of the functional components and the function-scalar hybrid structure.
	We describe the strategy at the $l$-th iteration. %For simplicity, we omit the iteration index and assume the observations have been residualized from previous steps (see Section~\ref{section:sub:residualization} for details). 
	The  PLS component direction is estimated by the unit-norm direction $\xi^{[l]} \in \widetilde{\mathbb{H}}$ that maximizes the squared empirical covariance, which quantifies the linear dependence between the PLS scores $\langle \widetilde{W}_1, \xi \rangle_{\mathbb{H}}, \ldots, \langle \widetilde{W}_n, \xi \rangle_{\mathbb{H}}$ and the responses $y_1, \ldots, y_n$, defined as
	
	\begin{equation}\label{def:squared_empirical_cov}
		\widehat{\textnormal{Cov}} 
		( \langle 
		\widetilde{W},
		\xi \rangle_{\mathbb{H}}, Y )
		:=
		\frac{1}{n} 
		\sum_{i=1}^n
		y_i
		\langle \widetilde{W}_i, \xi \rangle_{\mathbb{H}}
		.
	\end{equation}
	We denote the estimated PLS component direction as
	\begin{equation}\label{def:maximizer_squared_empirical_cov}
		\hat{\xi} 
		:= \arg \max_{\xi \in \widetilde{\mathbb{H}}
		}~\widehat{\textnormal{Cov}}^2 
		\bigl(
		\langle \widetilde{W}, \xi \rangle_{\mathbb{H}}, Y 
		\bigr)~s.t.~\| \xi  \|_\mathbb{H} = 1.
	\end{equation}
	%
	Here, $\hat{\xi} \in \widetilde{\mathbb{H}}$ is an ordered pair  expanded as:
	\begin{equation*}
		\hat{\xi}	= 
		\bigl(
		\hat{\xi}_1(t), \ldots, \hat{\xi}_K(t), \hat{\boldsymbol{\zeta}}
		\bigr)
		=
		\biggl(
		\sum_{m=1}^M \hat{\gamma}_{1m} \, b_m(t),
		\ldots, 
		\sum_{m=1}^M \hat{\gamma}_{Km} \, b_m(t), \hat{\boldsymbol{\zeta}}
		\biggr)
		,
	\end{equation*}
	where  $\hat{\boldsymbol{\zeta}}$ is the scalar part. Obtaining these coefficients is equivalent to solving the maximization problem   \eqref{def:maximizer_squared_empirical_cov}. The following proposition formulates this coefficients obtaining procedure as a generalized Rayleigh quotient:
	%
	\begin{proposition}\label{proposition:eigen_noregul}
		Let
		$
		(\mathbb{B}, \mathbb{B}^{\prime \prime}, \Theta, \mathbf{y})
		$
		denote the observed data defined in \eqref{def:problem_instance}.
		At the $l$-th iteration of the PLS algorithm, 
		the coefficients of the squared covariance  maximizer defined in \eqref{def:maximizer_squared_empirical_cov}, 
		is obtained as
		\begin{equation}	\label{eq: Unregularized generalized rayleigh quotient equation}
			\left(
			\hat{\gamma}_{11}, \ldots, \hat{\gamma}_{1M}
			, \ldots,
			\hat{\gamma}_{K1}, \ldots, \hat{\gamma}_{KM}
			, \hat{\boldsymbol{\zeta}}^\top
			\right)^\top
			=
			\arg \max_{\boldsymbol{\xi} \in \mathbb{R}^{MK+p}}   \boldsymbol{\xi}^\top V \boldsymbol{\xi}
			\quad \text{subject to} \quad \boldsymbol{\xi}^\top \mathbb{B} \boldsymbol{\xi} = 1.
		\end{equation}
		where
		\begin{equation}\label{def:V_matrix}
			V := \frac{1}{n^2} (\mathbb{B} \Theta^\top \mathbf{y})(\mathbb{B} \Theta^\top \mathbf{y})^\top \in \mathbb{R}^{(MK+p) \times (MK+p)}.
		\end{equation}
	\end{proposition}
	The proof of Proposition \ref{proposition:eigen_noregul}
	is provided in Appendix \ref{section:proof:proposition:linear_regul_system_unregularized}.
	
	\medskip
	\noindent
	\textit{Proposed regularized estimation procedure.}~ \label{sec: Regularization of PLS Components}
	Although Proposition \ref{proposition:eigen_noregul} offers an efficient way to estimate the PLS component direction $\hat{\xi}$, its functional components, $\hat{\xi}_1, \ldots, \hat{\xi}_K$, may not be smooth.  which complicates interpretation and can lead to overfitting and unstable predictions. 
	To address this, we propose a regularized extension that balances predictive performance with smoothness. Specifically, we penalize the roughness of each $\hat{\xi}_j$ using its integrated squared second derivative 
	\begin{equation}\label{def:roughness_penalty}
		\operatorname{PEN}(\hat{\xi}_j) := \int_0^1 \bigl\{ \hat{\xi}_j^{\prime\prime}(t) \bigr\}^2 dt
		.
	\end{equation}
	
	Instead of solely maximizing the squared empirical covariance
	$\widehat{\textnormal{Cov}}^2 
	\bigl(
	\langle \widetilde{W}, \xi \rangle_\mathbb{H}, Y 
	\bigr)
	$, 
	we incorporate this roughness penalty to simultaneously control the complexity of the estimated functional components.
	One possible approach is to extend the smoothed functional PCA framework of \citet{Rice1991} by modifying the objective in \eqref{def:maximizer_squared_empirical_cov} to
	\begin{equation*}
		\widehat{\textnormal{Cov}}^2 
		\bigl(
		\langle \widetilde{W}, \xi \rangle_{\mathbb{H}}, Y 
		\bigr)
		- \sum \limits_{j=1}^K \lambda_j 
		\operatorname{PEN}({\xi}_j)
		,
	\end{equation*}
	where the smoothing parameters $\{\lambda_j\}_{j=1}^K$ control the trade-off between maximizing covariance and penalizing roughness. 
	However, this approach  \citet{Rice1991} assumes that the functional predictors admit an orthogonal expansion in the $\mathbb{L}^2$ sense.
	
	
	To avoid the orthogonal basis assumption of  \citet{Rice1991}, we adopt the strategy of \citet{Silverman1996}, which replaces the standard orthonormality constraint with a weaker one based on a modified inner product that incorporates roughness. Accordingly, our estimation procedure at the $l$-th iteration, iteration index omitted and assuming the observations have been residualized in previous steps, solves the following optimization problem:
	\begin{equation}\label{def:maximizer_squared_empirical_cov_reg}
		\hat{\xi} 
		:= \arg \max_{\xi \in \widetilde{\mathbb{H}}
		}~\widehat{\textnormal{Cov}}^2 
		\bigl(
		\langle \widetilde{W}, \xi \rangle_{\mathbb{H}}, Y 
		\bigr)~s.t.~\| \xi  \|_\mathbb{H} +  \sum \limits_{j=1}^K \lambda_j 
		\operatorname{PEN}(\xi_K) = 1.
	\end{equation}
	Here, $\hat{\xi} \in \widetilde{\mathbb{H}}$ is an ordered pair  expanded as:
	\begin{equation*}
		\hat{\xi}	= 
		\bigl(
		\hat{\xi}_1(t), \ldots, \hat{\xi}_K(t), \hat{\boldsymbol{\zeta}}
		\bigr)
		=
		\biggl(
		\sum_{m=1}^M \hat{\gamma}_{1m} \, b_m(t),
		\ldots, 
		\sum_{m=1}^M \hat{\gamma}_{Km} \, b_m(t), \hat{\boldsymbol{\zeta}}
		\biggr)
		,
	\end{equation*}
	where  $\hat{\boldsymbol{\zeta}}$ is the scalar part.
	This formulation maximizes the squared covariance over a class of smooth functions.
	Obtaining these coefficients is equivalent to solving the maximization problem   \eqref{def:maximizer_squared_empirical_cov_reg}. The following proposition formulates this coefficients obtaining procedure as a generalized Rayleigh quotient:
	\begin{proposition}[Regularized estimation of PLS component direction]\label{proposition:eigen_regul}
		Let
		$
		(\mathbb{B}, \mathbb{B}^{\prime \prime}, \Theta, \mathbf{y})
		$
		denote the  data given at the $l$-th iteration, as defined in \eqref{def:problem_instance}.
		Recall from \eqref{def:V_matrix} that $
		V = n^{-2} (\mathbb{B} \Theta^\top \mathbf{y})(\mathbb{B} \Theta^\top \mathbf{y})^\top$ .
		Let 
		$\Lambda \in \mathbb{R}^{(MK+p) \times (MK+p)}$ be defined as:
		\begin{equation}\label{def:Lambda}
			\Lambda := \operatorname{blkdiag}(\lambda_1 I_M, \ldots, \lambda_K I_M, 0_{p \times p}),
			~\text{where}~\lambda_1, \ldots, \lambda_K \geq 0.
		\end{equation}
		Here, $0_{p \times p}$ denotes the $p \times p$ zero matrix.
		The coefficients of the squared covariance  maximizer defined in \eqref{def:maximizer_squared_empirical_cov_reg}, 
		are obtained as
		\begin{equation}	\label{eq: Regularized generalized rayleigh quotient equation}
			\left(
			\hat{\gamma}_{11}, \ldots, \hat{\gamma}_{1M}
			, \ldots,
			\hat{\gamma}_{K1}, \ldots, \hat{\gamma}_{KM}
			, \hat{\boldsymbol{\zeta}}^\top
			\right)^\top
			\hspace{-.7em}
			=
			\arg 
			\hspace{-.6em}
			\max_{\boldsymbol{\xi} \in \mathbb{R}^{MK+p}}   \boldsymbol{\xi}^\top V \boldsymbol{\xi}
			~~\text{s.t.}~~
			%
			% constraint
			\boldsymbol{\xi}^\top 
			\hspace{-.2em}
			(\mathbb{B} + \Lambda \mathbb{B}^{\prime \prime}) \boldsymbol{\xi} = 1.
		\end{equation}
		
	\end{proposition}
	The proof of Proposition \ref{proposition:eigen_regul} is provided in Appendix  \ref{section:proof:proposition:eigen_regul}.
	%orthonormality preview
	The constraint $	\boldsymbol{\xi}^\top (\mathbb{B} + \Lambda \mathbb{B}^{\prime \prime}) \boldsymbol{\xi} = 1$
	enforces the orthonormality of the estimated PLS component directions with respect to a modified inner product (see Section \ref{section:sub:geom} for details).
	The smoothing parameter $\lambda_k$  balances goodness of fit and smoothness in $\hat{\xi}_j$. Smaller $\lambda_k$ yields components that better fit the data but risks overfitting; setting $\lambda_k = 0$ recovers the unregularized solution in Proposition~\ref{proposition:eigen_noregul}. Larger $\lambda_k$ enforces greater smoothness, and in the limit $\lambda_k \to \infty$, $\hat{\xi}_j(t)$ approaches a linear form $a + bt$. In practice, both $\{\lambda_k\}$ and the number of components $L$ can be selected via cross-validation using a predictive criterion such as mean squared error.
	
	
	\medskip
	\noindent{\textit{Computation.}}~The generalized eigenproblem presented in Proposition~\ref{proposition:eigen_regul}  may be computationally unstable in practice. However, by leveraging the rank-one structure of the matrix 
	$V$
	Proposition \ref{proposition:linear_regul} derives a closed-form solution that requires only the solution of linear systems.
	\begin{proposition}[Closed-form solution]  
		\label{proposition:linear_regul}  
		Consider the optimization problem described in Proposition~\ref{proposition:eigen_regul}. Define the following quantities, which depend on the observed data but are not decision variables:  
		\[
		\mathbf{u}_j := B \Theta_j^\top \mathbf{y} \in \mathbb{R}^M \quad \text{for } j = 1, \ldots, K, \quad \text{and} \quad \mathbf{v} := \mathbf{Z}^\top \mathbf{y} \in \mathbb{R}^p.
		\]  
		Let
		\[
		q := \sum_{j=1}^K \mathbf{u}_j^\top (B + \lambda_j B^{\prime\prime})^{-1} \mathbf{u}_j + \mathbf{v}^\top \mathbf{v}.
		\]  
		
		Then the unique (up to sign) solution to the regularized maximization problem is given in closed form by  
		\[
		\hat{\boldsymbol{\gamma}}_j = \frac{1}{\sqrt{q}} (B + \lambda_j B^{\prime\prime})^{-1} \mathbf{u}_j \quad \text{for } j = 1, \ldots, K, \quad \text{and} \quad
		\hat{\boldsymbol{\zeta}} = \frac{1}{\sqrt{q}} \mathbf{v}.
		\]  
	\end{proposition}
	The proof of Proposition~\ref{proposition:linear_regul} is provided in Appendix~\ref{section:proof:proposition:linear_regul}. The expressions above involve solving linear systems for the functional and scalar components separately, followed by normalization by a common factor. Although the unnormalized coefficients are obtained independently, the normalization step couples the functional and scalar parts, allowing them to influence one another. This coupling enables the procedure to capture the correlation between the functional and scalar components of the PLS direction.
	
	
	
	
	
	\subsubsection{Iterative step 2: residualization via hybrid-on-scalar regression} \label{section:sub:residualization}
	The $l$-th iteration's second step involves residualization of both predictors and responses. We first compute the individual PLS score:  
	\begin{equation}\label{def:plsscore}
		\hat{\rho}^{[l]}_i := \langle \widetilde{W}_i^{[l]}, \, \hat{\xi}^{[l]} \rangle_{\mathbb{H}},
	\end{equation}
	using the estimated PLS component direction $\widehat{\xi}^{[l]}$ obtained from Propositions \ref{proposition:eigen_regul} and \ref{proposition:linear_regul}. 
	Since   $\widetilde{W}_i^{[l]}$ are assumed to have a sample mean of zero, these PLS scores will also have a sample mean of zero.
	To obtain the $(l+1)$-th  iteration's responses and hybrid predictors, we regress the $(l)$-th iteration's responses and hybrid predictors on  
	these PLS scores by least squares and then residualize. 
	Specifically,
	the $(l+1)$-th predictor is computed as a residual of hybrid-on-scalar linear regression model:
	\begin{equation*}
		\widetilde{W}_i^{[l]}  =
		\widehat{\rho}_i^{[l]}
		\hat{\delta}^{[l]} + \epsilon_i,
	\end{equation*}
	where $\hat{\delta}^{[l]} \in \widetilde{\mathbb{H}}$ is the regression coefficient.
	In the same spirit as the PLS component direction estimation step,
	rather than treating the hybrid object as a long vector of concatenated function evaluations at time points and scalar vectors,
	we employ a basis expansion approach to fit the entire hybrid object in one step.
	Therefore, our method is computationally efficient, and applicable for dense or irregular functional data.
	Consequently, $\delta^{[l]}$ is obtained by minimizing a  
	least squares criterion: impossible. Smoothness of $\beta$ seems to soley rely on the smoothness of $\xi$. See Section \ref{section:sub:regression_coeff}.
	\begin{equation}\label{def:argmin_pensse}
		\hat{\delta}^{[l]} := \arg \min_{\delta \in \widetilde{\mathbb{H}}} \sum_{i=1}^n 
		\| \widetilde{W}_i^{[l]} - \hat{\rho}_i^{[l]}\, \delta\|_{\mathbb{H}}^2.
	\end{equation}
	On the other hand, the $(l+1)$-th response   is computed as a residual of a scalar-on-scalar linear regression model:
	\begin{equation*}
		Y_i^{[l]}  =
		\hat{\nu}^{[l]}   	\widehat{\rho}_i^{[l]} + \epsilon_i.
	\end{equation*}
	The following proposition demonstrates that this residualization step can be performed simply, analogous to scalar PLS.
	\begin{lemma}[Closed-form  solution]\label{proposition:closed_form_orthgonalization}
		Let us denote
		$
		\hat{\boldsymbol{\rho}}^{[l]} := (\hat{\rho}^{[l]}_1, \ldots, \hat{\rho}^{[l]}_n)^\top.
		%\quad\text{and} \quad		\bar{\rho}^{[l]} := (\hat{\boldsymbol{\rho}}^{[l]})^\top \hat{\boldsymbol{\rho}}^{[l]}.
		$
		The $(l+1)$-th iteration's  predictors and responses are computed as
		\begin{equation*}
			\widetilde{W}_i^{[l+1]} 
			:= 	\widetilde{W}_i^{[l]}  - 
			\widehat{\rho}_i^{[l]}
			\hat{\delta}^{[l]},~\text{where}~
			\delta^{[l]}
			:= 
			\frac{1}{\| 	\hat{\boldsymbol{\rho}}^{[l]}\|_2^2}
			\sum_{i=1}^n 
			\widehat{\rho}_i^{[l]}
			\widetilde{W}_i^{[l]},
			%%%%%%%%%%% 
			,
		\end{equation*} 	
		and
		\begin{equation}\label{algorithm_step:residualization_predictor}
			Y_i^{[l+1]} = Y_i^{[l]} - 
			\hat{\nu}^{[l]} 
			\widehat{\rho}_i^{[l]},~\text{where}~\hat{\nu}^{[l]} :=  
			\frac{
				\mathbf{y}^{[l] \top }  \hat{\boldsymbol{\rho}}^{[l]}
			}{
				\| \hat{\boldsymbol{\rho}}^{[l]} \|_2^2
			}.
		\end{equation}
	\end{lemma}
	Proof of Lemma \ref{proposition:closed_form_orthgonalization} is provided in Appendix \ref{section:proof:proposition:closed_form_orthgonalization}.
	Since $\widetilde{W}_i^{[l]}$ and $Y_i^{[l]}$ are assumed to have a sample mean of zero, their respective residuals, $\widetilde{W}_i^{[l+1]}$ and $Y_i^{[l+1]}$, also maintain a zero sample mean.
	
	
	
	
	
	\subsection{Final step: estimating  the hybrid regression coefficient}
	\label{section:sub:regression_coeff}
	The hybrid regression coefficient $\beta$ in model \eqref{eq: Hybrid functional model} can be written 
	as a linear combination of PLS directions:
	\begin{lemma}\label{lemma:recursive}
		Let use define
		$
		\widehat{\iota}^{[1]} := \widehat{\xi}^{[1]}.
		$
		For $l \ge 2$, we recursively   define:
		\begin{equation*}
			\widehat{ \iota}^{[l]} = \widehat{\xi}^{[l]} - \sum_{u=1}^{l-1} \langle \widehat{ \delta}^{[u]}, \widehat{\xi}^{[l]} \rangle_{\mathbb{H}} \widehat{ \iota}^{[u]}. 
		\end{equation*}
		Then we have:
		\begin{equation*}
			\widehat{\rho}_i^{[l]}
			=
			\langle  W_i^{[l]}, \widehat{\xi}^{[l]} \rangle_\mathbb{H}
			=
			\langle  W_i, \widehat{ \iota}^{[l]} \rangle_\mathbb{H}.
		\end{equation*}
	\end{lemma}
	Proof of Lemma \ref{lemma:recursive} is provided in Appendix \ref{section:proof:lemma:recursive}.
	
	Next,  \eqref{algorithm_step:residualization_predictor} leads to the following model:
	\begin{equation*}
		Y_i = \sum \limits_{l=1}^L \widehat{\nu}^{[l]} \widehat{\rho}_i^{[l]} + \epsilon_i.
	\end{equation*}
	This model lets us to express $Y_i$ as:
	\begin{equation*}
		Y_i = \sum \limits_{l=1}^L \widehat{\nu}^{[l]} \langle W_i^{[l]}, \widehat{\xi}^{[l]} \rangle_\mathbb{H} +\epsilon_i 
		= 
		\bigl \langle W_i, \sum \limits_{l=1}^L \widehat{\nu}^{[l]} \widehat{ \iota }^{[l]}
		\bigr \rangle_\mathbb{H}+\epsilon_i,
	\end{equation*}
	which, given the uniqueness of $ \beta $, leads to
	\begin{equation*}
		\widehat{ \beta} = \sum \limits_{l=1}^L\widehat{\nu}^{[l]} \widehat{ \iota }^{[l]}
		\sum_{l=1}^L \left( \widehat{\nu}^{[l]} - \sum_{k=l+1}^L \widehat{\nu}^{[k]} \langle \widehat{\delta}^{[l]}, \widehat{\xi}^{[k]} \rangle_{\mathbb{H}} \right) \widehat{\xi}^{[l]}.
	\end{equation*}
	Thus the number of PLS components $L$ to be estimated can be chosen by cross-validation.
	
	\begin{algorithm} 
		\caption{Hybrid partial least squares regression}\label{alg:hybrid_pls}
		\begin{algorithmic}[1]
			\State 	\textbf{Initialize:} $(\mathbb{B}, \mathbb{B}^{\prime\prime}, \Theta^{[1]}, \mathbf{y}^{[1]})$ as the data objects after basis expansion, following Section \ref{sec: finite basis approximation}.
			
			\State   $\widetilde{W}_1^{[1]}, \ldots, \widetilde{W}_n^{[1]}, Y_1^{[1]}, \ldots, Y_n^{[1]} \leftarrow$ standardized versions of $W_1, \ldots, W_n, Y_1, \ldots, Y_n$, following Section \ref{subsec: Data Preprocessing}
			\For{$l = 1, 2, \ldots, L$}
			\\
			\textbf{PLS direction and score estimation  (Proposition \ref{proposition:linear_regul}): }
			\State $\mathbf{u}_j^{[l]} \leftarrow B \Theta_j^{[l]\top} \mathbf{y}^{[l]}, ~j = 1, \ldots, K$
			\State $\mathbf{v}^{[l]} \leftarrow \mathbf{Z}^{[l]\top} \mathbf{y}^{[l]} $
			\State $	q^{[l]}  \leftarrow \sum_{j=1}^K \mathbf{u}_j^{[l]\top} (B + \lambda_j B^{\prime\prime})^{-1} \mathbf{u}_j^{[l]} + \mathbf{v}^{[l]\top} \mathbf{v}^{[l]}$
			\State $( \hat{\gamma}_{j1}^{[l]} , \ldots, \hat{\gamma}_{jM}^{[l]} )^\top  \leftarrow \frac{1}{\sqrt{q}} (B + \lambda_j B^{\prime\prime})^{-1} \mathbf{u}_j^{[l]} , ~j = 1, \ldots, K$
			\State $\hat{\boldsymbol{\zeta}}^{[l]}  \leftarrow \frac{1}{\sqrt{q}} \mathbf{v}^{[l]} $
			\State $	\hat{\xi}^{[l]}	\leftarrow 
			\biggl(
			\sum_{m=1}^M \hat{\gamma}_{1m}^{[l]} \, b_m(t),
			\ldots, 
			\sum_{m=1}^M \hat{\gamma}_{Km}^{[l]} \, b_m(t), \hat{\boldsymbol{\zeta}}^{[l]}
			\biggr)
			$
			\Comment{PLS direction} 
			\State 
			$
			\widehat{\rho}_i^{[l]} \leftarrow \langle \hat{\xi}^{[l]},   \tilde{W}^{[l]}_i \rangle, 
			~i=1,  \ldots, n
			$ \Comment{PLS score}
			\\
			\textbf{Residualization (Proposition \ref{proposition:closed_form_orthgonalization}):}
			\State $\nu^{[l]} 
			\leftarrow
			\frac{
				\sum_{i=1}^n Y_i^{[l]} \widehat{\rho}_{i}^{[l]}}{
				\sum_{i=1}^n \widehat{\rho}_{i}^{[l]2}
			}$ \Comment{Least squares estimate} 
			%
			\State $ Y_i^{[l+1]} \leftarrow Y_i^{[l]} - \nu^{[l]}\widehat{\rho}_i^{[l]}~i=1,  \ldots, n$
			%
			\State $ \widehat{ \delta }^{[l]}  \leftarrow \frac{1}{\sum_{i=1}^n \widehat{\rho}_{i}^{[l]2}}\sum_{i=1}^n  \widehat{\rho}_{i}^{[l]}\widetilde{W}_i^{[l]} $ \Comment{Least squares estimate} 
			\label{alg:step:scalar_pls_residualization}
			\State $\widetilde{W}_i^{[l+1]}  \leftarrow \widetilde{W}_i^{[l]} -   \widehat{\rho}_{i}^{[l]}  \widehat{ \delta}^{[l]} $
			\EndFor
			\textbf{Regression coefficient estimation (Section \ref{section:sub:regression_coeff}):}
			\State
			$\widehat{ \iota }^{[1]} \leftarrow \widehat{ \xi }^{[1]}$
			\For{$l =  2, \ldots, L$}
			\State $\widehat{ \iota }^{[l]} \leftarrow \widehat{  \xi }^{[l]} - \sum_{u=1}^{l-1} \langle \widehat{ \delta }^{[u]}, \widehat{ \xi }^{[l]} \rangle  \widehat{ \iota }^{[u]}$  
			\EndFor
			$	\widehat{ \beta } \leftarrow \sum \limits_{l=1}^L\widehat{\nu}^{[l]} \widehat{ \iota}^{[l]}$ \Comment{regression coefficient estimate}
			\State \textbf{Output:} the  regression coefficient estimate $	\widehat{ \beta }$
		\end{algorithmic}
	\end{algorithm}
	
	
	\section{Properties of the hybrid PLS}
	\label{section:main:properties_of_our_method}
	This section provides the mathematical properties that support the algorithm suggested in Section \ref{section:main:our_algorithm}.
	Section \ref{section:sub:tucker} shows that the core optimization problem for the Partial Least Squares (PLS) direction estimation step, presented in Proposition \ref{proposition:eigen_noregul}, is well-defined under mild conditions.
	Section \ref{section:sub:geom} demonstrates that our algorithm preserves the core properties of PLS, namely the orthonormality of the derived directions and the orthogonality of the scores.
	
	\subsection{ Tucker's Criterion  }\label{section:sub:tucker} 
	We now derive the population analogue of Algorithm~\ref{alg:hybrid_pls} and establish its key mathematical properties, namely the orthogonality of the PLS directions and scores, as well as convergence to the true response. At the population level, where the random objects are fully observable, we may drop the sample index $i$ and simply work with $Y$
	and
	$W = (X_1, \ldots, X_K, Z_1, \ldots, Z_p)$. Throughout this section we assume 
	$\mathbb{E}[Y] = 0 \in \mathbb{R}$ and $\mathbb{E}[W] = 0 \in \mathbb{H}$.
	
	The core optimization problem in PLS is formulated in terms of maximizing the squared cross-covariance. At the population level, this quantity is naturally characterized by cross-covariance operators, which we define next. For notational simplicity, we suppress the iteration index $l$, with the understanding that for $l=1$ the predictors correspond to the original variables, while for $l \geq 2$ they represent the residualized versions.
	\begin{definition}\label{def:cross_cov_terms}
		The covariance operator of the hybrid predictor $W$ is denoted as $\Sigma_{W} := \mathbb{E}[W \otimes W]$, so that for $u,v \in \mathbb{H}$, we have
		\[
		\Sigma_{W} u = \mathbb{E}[
		\langle W, u\rangle_{\mathbb{H}}W
		],
		\quad
		\langle \Sigma_{W} u, v \rangle_{\mathbb{H}}
		= \mathbb{E}[\langle W,u \rangle_{\mathbb{H}} \, \langle W,v \rangle_{\mathbb{H}}].
		\]
		
		The cross-covariances between the response $Y$ and the predictors are
		\[
		\sigma_{YX}
		:=
		\bigl(\mathbb{E}[Y X_{1}],\ldots,\mathbb{E}[Y X_{K}]\bigr) \in \mathcal{F}^{K},
		\quad
		\sigma_{YZ} 
		:=
		\bigl(\mathbb{E}[Y Z_{1}],\ldots,\mathbb{E}[Y Z_{p}]\bigr)^\top \in \mathbb{R}^p.
		\]
		
		The cross-covariance between $Y$ and the hybrid predictor $W$ is then
		\[
		\Sigma_{YW} := \mathbb{E}[ Y W ] = \bigl( \sigma_{YX}, \, \sigma_{YZ} \bigr) \in \mathbb{H}.
		\]
	\end{definition}
	At the population level, Algorithm \ref{alg:hybrid_pls} without regularizaiton can be expressed as follows.
	Let $W^{[1]} := W$ and $Y^{[1]} := Y$.
	For any   integer $l \geq 2$,
	to be the residuals of the 
	
	where 
	
	and
	
	which is attained by the leading eigenfunction of the operator 
	$\mathcal{U}^{[l-1]}$, defined analogously to Lemma~\ref{lemma:composite_cross_cov} 
	using $W^{[l-1]}$ and $Y^{[l-1]}$, provided the conditions in 
	Lemma~\ref{lemma:cross_cov_functional} are satisfied.
	\begin{definition}[Population PLS]\label{def:population_pls}
		The population analogue of the PLS optimization problem \eqref{scalar_PLS_sample} in the hybrid space is formulated as the constrained generalized eigenproblem:
		\begin{equation}\label{const_eigen_pop}
			\xi^{[l]} = \arg\max_{h \in \mathbb{H}}
			\operatorname{Cov}^2(\langle W, h \rangle_{\mathbb{H}}, Y),~s.t.~\|h\|_{\mathbb{H}}=1, \langle h, \Sigma_W \, \xi^{[j]} \rangle_{\mathbb{H}} = 0,\, j < l.
		\end{equation}
	\end{definition}
	As in scalar case, this problem can be solved by an iterative algorithm, and yields orthogonal scores:
	\begin{proposition}[Properties of population PLS]
		\label{prop:residualization_equiv_eigen}
		The solution of optimization problem \eqref{const_eigen_pop} of Definition \ref{def:population_pls} is equivalent to that of population NIPALS (Algorithm \ref{alg:population_pls}).  For every \(s\ge 1\) and every \(j\le s-1\), we have
		$
		\mathbb{E}[\rho^{[j]}\rho^{[s]}]=0
		$
		and
		$
		\mathbb{E}[Y^{[s]}\rho^{[j]}]=0.
		$
		
	\end{proposition}
	Proof of Proposition \ref{prop:residualization_equiv_eigen} is provided in Appendix \ref{section:proof:prop:residualization_equiv_eigen}.
	\begin{algorithm} 
		\caption{Population NIPALS}\label{alg:population_pls}
		\begin{algorithmic}[1]
			\State
			$(W^{[1]}, Y^{[1]} \leftarrow (W, Y)$
			\For{$l = 1, 2, \ldots, L$}
			\State
			$
			\xi^{[l]}
			\leftarrow
			\arg\max_{h \in \mathbb{H}} \operatorname{Cov}^2(\langle W^{[l]}, h \rangle_{\mathbb{H}}, Y^{[l]})~s.t.~ \|h\|_{\mathbb{H}} = 1,
			$\label{alg:line:pls_direction} \Comment{PLS direction}
			%%%%%%%%%
			\State 
			$
			\rho^{[l]} \leftarrow \langle \xi^{[l]}, W^{[l]} \rangle 
			$ \Comment{PLS score}
			%%%%%%%%%%
			\State
			$
			\delta^{[l]}
			\leftarrow
			\frac{1}{\mathbb{E}[ (\rho^{[l]})^2 ]}  \mathbb{E}[W^{[l]} \rho^{[l]}]
			$\label{population_delta}
			\Comment{Linear regression of 
				$W^{[l]}$
				on $\rho^{[l]}$}
			%%%%%%%%%%%
			\State
			$
			\nu^{[l]}
			\leftarrow
			\frac{1}{\mathbb{E}[ (\rho^{[l]})^2 ]}
			\mathbb{E}[Y^{[l]} \rho^{[l]}]
			$
			\Comment{
				Linear regression of 
				$Y^{[l]} $
				on $\rho^{[l]}$
			}
			%%%%%%%%%%%%%%%%%%%
			\State 
			$ W^{[l+1]} \leftarrow W^{[l]} - \rho^{[l]} \, \delta^{[l]}$
			\label{population_predictor_residualize}
			\Comment{
				Residualized predictor
			}
			\State
			$ Y^{[l]} \leftarrow Y^{[l-1]} - \nu^{[l-1]} \, \rho^{[l-1]}$
			\Comment{
				Residualized response
			}
			\EndFor
			\State \textbf{Output:}
			PLS directions $\xi^{[1]}, \ldots, \xi^{[L]}$
		\end{algorithmic}
	\end{algorithm}
	
	We show that problem \eqref{const_eigen_pop} and Algorithm \ref{alg:population_pls} are well-defined under mild conditions by  establishing that line \ref{alg:line:pls_direction} of Algorithm \ref{alg:population_pls} is well-defined under mild conditions. For clarity, we define the relevant operators while omitting the iteration index for the residualized predictors and responses.
	\begin{lemma}\label{lemma:cross_cov_functional}
		Define the operator $
		\mathcal{C}_{YW}$ such that for   any $h=(f_1, \ldots, f_K, \mathbf{v}) \in \mathbb{H}$,
		\begin{equation*} \mathcal{C}_{YW} h := \mathbb{E} \left[ \langle W, h \rangle_\mathbb{H} Y \right] = \langle \Sigma_{YW}, h \rangle_\mathbb{H} \in \mathbb{R}.
		\end{equation*} 
		If 
		$\sigma_{YX}$ and $\sigma_{YZ}$ are bounded in the sense that
		there exist finite constants $Q_1$ and $Q_2$ such that
		\begin{align}\label{condition_for_compact}
			\max \limits_{k=1, \ldots, K} \sup \limits_{t \in [0,1]} \mathbb{E}
			[Y X_{k}](t)^2 < Q_1 \quad \textnormal{and} \quad  \max_{r=1, \ldots, p} \mathbb{E}
			[Y Z_{r}]^2 < Q_2,
		\end{align}
		the operator $\mathcal{C}_{YW}$ is a compact operator.
	\end{lemma}
	Proof of Lemma \ref{lemma:cross_cov_functional}
	is provided in Appendix \ref{section:proof:lemma:cross_cov_functional}.
	\begin{lemma}\label{lemma:cross_cov_operator}
		Define $\mathcal{C}_{WY} = \mathbb{E}[Y \otimes W]$,
		such that for any  $d \in \mathbb{R}$,
		%%%%
		\begin{align*}		
			\mathcal{C}_{WY} d 
			:= 
			\mathbb{E} 
			\bigl[
			\langle Y, d \rangle W 
			\bigr] 
			= 
			\mathbb{E}
			[ Y W d ] =  d \, \Sigma_{Y W} \in \mathbb{H}.
		\end{align*}
		Then	$\mathcal{C}_{WY}$ is an adjoint operator of $\mathcal{C}_{YW}$. That is, $\mathcal{C}_{WY} = \mathcal{C}_{YW}^\ast$.
	\end{lemma}	
	The proof of Lemma \ref{lemma:cross_cov_operator} is provided in Appendix \ref{section:proof:lemma:cross_cov_operator}.
	
	\begin{lemma}[Composite cross-covariance operator]\label{lemma:composite_cross_cov}
		Define
		$\mathcal{U} := \mathcal{C}_{WY} \circ \mathcal{C}_{YW}: \mathbb{H} \rightarrow \mathbb{H}$ as an operator which performs the following mapping:
		\begin{equation*}
			\mathcal{U} h 
			= 
			\mathcal{C}_{WY} ( \mathcal{C}_{YW} h) 
			=
			\mathcal{C}_{YW} (\langle \Sigma_{YW}, h \rangle_\mathbb{H}) 
			=  
			\Sigma_{YW} \, \langle \Sigma_{YW}, h \rangle_\mathbb{H} 
			= 
			(\Sigma_{YW} \otimes \Sigma_{YW}) h.
		\end{equation*}
		In other words, $\mathcal{U} = \Sigma_{YW} \otimes \Sigma_{YW}$.
		Then $\mathcal{U}$
		is a self-adjoint and positive-semidefinite operator.
		Under the conditions of Lemma \ref{lemma:cross_cov_functional}, $\mathcal{U}$ is a compact operator.
	\end{lemma}
	The proof of Lemma \ref{lemma:composite_cross_cov} is provided in Appendix \ref{section:proof:lemma:composite_cross_cov}.
	By the Hilbert-Schmidt theorem (e.g., Theorem 4.2.4 in \citealp{hsingTheoreticalFoundationsFunctional2015}), Lemma \ref{lemma:composite_cross_cov} guarantees the existence of a complete orthonormal system of eigenfunctions $\{\xi_{(u)}\}_{u \in \mathbb{N}}$ of $\mathcal{U}$ in $\mathbb{H}$ such that $\mathcal{U} \xi_{(u)} = \kappa_{(u)}  \xi_{(u)}$, where $\{\kappa_{(u)}\}_{u \in \mathbb{N}}$ are the corresponding sequence of eigenvalues that goes to zero as $u \rightarrow \infty$, that is, $\kappa_{(1)} \ge \kappa_{(2)} \ge \cdots \ge 0$.
	
	The following theorem presents population-level PLS and Tucker's Criterion for our scalar-on-hybrid regression model and ensures that line \ref{alg:line:pls_direction} of Algorithm \ref{alg:population_pls} is well-defined under mild conditions.
	\begin{theorem}[Tucker's criterion]\label{theorem:population_PLS} Under the conditions of Lemma \ref{lemma:cross_cov_functional}, the constrained maximum
		\begin{equation*} 
			\max \limits_{\substack{ \Vert \xi \Vert_\mathbb{H}=1}} \operatorname{Cov}^2 \left(\langle W , \, \xi \rangle_\mathbb{H}, Y \right)
		\end{equation*} 
		is attained by the eigenfunction associated with the largest eigenvalue of the operator $\mathcal{U}$.
	\end{theorem}
	The proof of Theorem \ref{theorem:population_PLS} is provided in Appendix \ref{section:proof:theorem:population_PLS}.
	
	Finally, the following theorem states that the fitted values from hybrid PLS converge to those of ordinary linear regression in the mean-squared sense. This shows that hybrid PLS, as a dimension-reduction technique, can effectively capture the predictive power of a full linear model when a sufficient number of components is used.
	\begin{theorem}[$L^2$ Convergence of Hybrid PLS at Population Level]\label{thm:hybrid_pls_convergence}
		Let $Y_{\mathrm{PLS}, L} = \sum_{l=1}^L \nu^{[l]} \rho^{[l]}$ be the Hybrid PLS fitted value with $L$ components. Then, the Hybrid PLS fitted value converges to $Y$ in the mean-squared sense:
		\begin{equation*}
			\lim_{L \to \infty} \mathbb{E}\left[ \| Y_{\mathrm{PLS},L } - Y  \|_2^2 \right] = 0
		\end{equation*}
	\end{theorem}
	The proof of Theorem \ref{thm:hybrid_pls_convergence} is provided in Appendix \ref{section:proof:thm:hybrid_pls_convergence}.
	
	\subsection{Geometric properties}\label{section:sub:geom}
	A fundamental property of partial least squares,
	as seem in Definition \ref{def:population_pls} and Proposition \ref{prop:residualization_equiv_eigen},
	is that between iterations, its derived directions are orthonormal and PLS scores are orthogonal. Our regularized estimates preserve this property, with respect to a modified inner product that incorporates the roughness penalty, defines as follows:
	\begin{definition}[Roughness-sensitive inner product]\label{def:hybrid_inner_product_roughness}
		Given two hybrid predictors $W_1 = (X_{11}, \ldots, X_{1K}, \mathbf{Z}_1)$ and $W_2 = (X_{21}, \ldots, X_{2K}, \mathbf{Z}_2)$, both elements of $\mathbb{H}$ as defined in Definition \ref{def:hybrid_predictor}, and a roughness penalty matrix $\Lambda = \operatorname{blkdiag}(\lambda_1 I_M, \ldots, \lambda_K I_M, 0_{p \times p})$, the roughness-sensitive inner product between $W_1$ and $W_2$ is defined as:
		\begin{equation}		\label{eq: hybrid inner product_roughness}
			\langle W_1, W_2\rangle_{\mathbb{H}, \Lambda}
			:=
			\sum \limits_{k=1}^K \int_0^1 X_{1k}(t) X_{2k}(t) \, dt
			+
			\sum \limits_{k=1}^K \lambda_k \int_0^1 X^{\prime \prime}_{1k}(t) X^{\prime \prime}_{2k}(t) \, dt
			+
			\mathbf{Z}_1^\top \mathbf{Z}_2.
		\end{equation}
	\end{definition}
	Based on this inner product, the following proposition states that the PLS component directions estimated from Proposition  \ref{proposition:eigen_regul} are orthonormal.
	\begin{proposition}[Orthonormality of estimated PLS component directions]\label{proposition: modified orthnormality of PLS components}
		The PLS component directions $
		\widehat{\xi}^{[1]}, \widehat{\xi}^{[2]}, \ldots, \widehat{\xi}^{[L]}
		$, estimated via Proposition \ref{proposition:eigen_regul} with a roughness penalty matrix $\Lambda = \operatorname{blkdiag}(\lambda_1 I_M, \ldots, \lambda_K I_M, 0_{p \times p})$, are mutually orthonormal with respect to the inner product $\langle \cdot, \cdot \rangle_{\mathbb{H}, \Lambda}$. That is,
		\begin{equation*}
			\langle \widehat{\xi}^{[l_1]}, \widehat{\xi}^{[l_2]} \rangle_{\mathbb{H}, \Lambda}
			= \mathbbm{1}(l_1 = l_2), \quad l_1, l_2 = 1, \ldots, L.
		\end{equation*}
	\end{proposition}
	The proof of Proposition \ref{proposition: modified orthnormality of PLS components} is provided in Appendix \ref{section:proof:proposition: modified orthnormality of PLS components}.
	
	The next proposition states that the vectors of estimated PLS scores for different iteration numbers are mutually orthogonal.
	\begin{proposition}	\label{proposition: orthnormality of PLS scores}
		Recall from Lemma \ref{proposition:closed_form_orthgonalization} that   $\widehat{\boldsymbol{\rho}}^{[l]}$ denote the $n$-dimensional vector whose elements consist of the $l$-th estimated PLS scores ($l=1, \ldots, L$) of $n$ observations.
		The vectors $\widehat{\boldsymbol{\rho}}^{[1]}, \widehat{\boldsymbol{\rho}}^{[2]}, \ldots, \widehat{\boldsymbol{\rho}}^{[L]}$  are mutually orthogonal in the sense that
		\begin{equation*}
			\widehat{\boldsymbol{\rho}}^{[l_1] \top}  \widehat{\boldsymbol{\rho}}^{[l_2] }= 0 \quad \textnormal{for} \quad \; l_1,l_2 \in \{1, \ldots, L\}, \; l_1 \ne l_2. 
		\end{equation*}
	\end{proposition}
	The proof of Proposition \ref{proposition: orthnormality of PLS scores} is provided in Appendix \ref{section:proof:proposition: orthnormality of PLS scores}.
	
	\section{Simulation studies}
	To evaluate the superiority of our method under complex dependency structures
	{\textemdash}
	specifically, dependencies among functional predictors,
	among scalar predictors,
	and between scalar and functional predictors
	{\textemdash}
	
	\subsection{Synthetic data}
	To generate functional predictors, we use a matrix-normal distribution, also known as a Kronecker-separable covariance model. For each of $i=1, \ldots,  n$, we create a $3 \times 100$ matrix, where each row represents the $i$th observation of a functional predictor with 100 evaluation points. This method gives us precise control over  cross-modal (between-row) dependence   and 
	smoothness of the curves (between-column dependence). 
	\begin{equation*}
		\mathbf{X} \sim \mathcal{MN}_{m \times n}(\mathbf{M}; \mathbf{R}, \mathbf{C}),
	\end{equation*}
	and its log-density is given by
	\[
	\log p(\mathbf{X} \mid \mathbf{M}, \mathbf{R}, \mathbf{C}) = -\frac{mn}{2} \log(2\pi) - \frac{m}{2} \log |\mathbf{C}| - \frac{n}{2} \log |\mathbf{R}| - \frac{1}{2} \mathrm{Tr}\left[ \mathbf{C}^{-1} (\mathbf{X} - \mathbf{M})^\top \mathbf{R}^{-1} (\mathbf{X} - \mathbf{M}) \right].
	\]
	The key insight behind Kronecker separability is that if
	$
	\mathbf{Y} \sim \mathcal{MN}(\mathbf{M}, \mathbf{R}, \mathbf{C}),
	$
	then its vectorized form follows a multivariate normal distribution:
	$
	\mathrm{vec}(\mathbf{Y}) \sim \mathcal{N}(\mathrm{vec}(\mathbf{M}), \mathbf{C} \otimes \mathbf{R}),
	$
	where $\otimes$ denotes the Kronecker product and $\mathrm{vec}$ is the vectorization operator.
	
	We developed a data generation framework leveraging the matrix-normal distribution, an approach previously suggested by our advisor. The goal was to create a dataset with a specific, controllable correlation structure between functional and scalar predictors.
	\begin{comment}
	The synthetic dataset consisted of n 
	sample
	​
	matrices, each of size 3×n 
	col
	​
	, where n 
	col
\end{comment}	​
	corresponds to the number of evaluation points for each curve. To introduce a structured dependence within each row, the row covariance matrix was set to mimic an AR(1) process, which is a special case of a Gaussian process. This ensures that each row exhibits the smooth, curve-like characteristics of a functional observation.
	
	The first two rows of each matrix were treated as functional data. The third row was converted into a set of scalar predictors. This was accomplished by partitioning the third row into blocks and then averaging the values within each block. For instance, if the third row was (1,3,2,4,3,4) and we wanted to generate three scalar predictors, we would partition the row into (1,3), (2,4), and (3,4), yielding the scalar values 2.0,3.0, and 3.5.
	
	\begin{comment}
	\begin{figure}[ht!]
		\centering
		\begin{subfigure}[b]{0.45\textwidth}
			\centering
			\includegraphics[width=\textwidth]{image/cor_strong.pdf}
			\caption{Correlation Matrix - Strong Dependency}
			\label{fig:cor_strong}
		\end{subfigure}
		\hfill
		\begin{subfigure}[b]{0.45\textwidth}
			\centering
			\includegraphics[width=\textwidth]{image/cor_weak.pdf}
			\caption{Correlation Matrix - Weak Dependency}
			\label{fig:cor_weak}
		\end{subfigure}
		
		\vspace{0.5cm}  % Adjust vertical spacing
		
		\begin{subfigure}[b]{0.45\textwidth}
			\centering
			\includegraphics[width=\textwidth]{image/graph_strong.pdf}
			\caption{Graph Structure - Strong Dependency}
			\label{fig:graph_strong}
		\end{subfigure}
		\hfill
		\begin{subfigure}[b]{0.45\textwidth}
			\centering
			\includegraphics[width=\textwidth]{image/graph_weak.pdf}
			\caption{Graph Structure - Weak Dependency}
			\label{fig:graph_weak}
		\end{subfigure}
		
		\caption{Comparison of Correlation Matrices and Graph Structures under Strong and Weak Dependencies.}
		\label{fig:correlation_graph_comparison}
	\end{figure}
\end{comment}
	\begin{comment}
		The goal of this simulation is to:
		\begin{enumerate}
			\item Utilize a graphical model to introduce a complex dependence structure.
			\item Demonstrate that our method outperforms the penalized functional regression approach by \citet{goldsmithPenalizedFunctionalRegression2011}.
			\item Show that our method exhibits superior performance as correlation strength increases.
		\end{enumerate}
		
		Our previous setup, presented in Section \ref{simul:meeting_feb}, appeared to achieve these objectives. However, we identified several limitations:
		\begin{enumerate}
			\item The resulting precision matrix was singular, requiring us to scale up the diagonal elements to compute its inverse. This adjustment disrupted the original AR(1) structure of the functional component, making it significantly less smooth.
			\item Increasing conditional correlation did not consistently enhance the superiority of our method.
		\end{enumerate}
		To address these issues, I will explore alternative functional graphical model settings that exhibit a strong dependence structure—one substantial enough to be detected by statistical methodologies and validated by top statistical journal papers. I will evaluate the performance of penalized functional regression on these data-generating models.
		Among them, I will select three to four settings for comparison against other methods.
	\end{comment}
	Building on the functional graphical model simulation of \citet{zhuBayesianGraphicalModels2016}, we generate a mixed graphical model with five nodes, described as follows:  
	\begin{itemize}
		\item Nodes F1 and F2: Two functional predictors modeled as Gaussian processes using a truncated Karhunen-Loève expansion, where the eigenbasis consists of Fourier basis functions with a fixed number of basis functions, \(M = 9\).  
		\item  Nodes S1, S2, and S3: Three scalar predictors, each following an \(s\)-dimensional multivariate normal distribution. Unlike the functional predictors, these scalar predictors are modeled directly without basis expansion.  
	\end{itemize}
	To capture dependencies among predictors, we introduce a graph structure that governs their conditional correlations. We consider two types of graph structures: a weakly connected graph and a strongly connected graph. In the Gaussian process framework, the precision matrix \(\mathbf{R}_0^{-1}\) encodes conditional independence relationships, while its inverse, \(\mathbf{R}_0\), represents marginal covariances. This structure extends to a blockwise correlation matrix \(\mathbf{R} \in \mathbb{R}^{(2M + 3s) \times (2M + 3s)}\), where off-diagonal blocks represent correlations between FPC scores and scalar predictor values.  
	Each block \((i, j)\) of \(\mathbf{R}\) is given by \((R_0)_{ij} \mathbf{I}_{M_i, M_j}\), where \(\mathbf{I}_{M_i, M_j}\) is a rectangular identity matrix. Here, \(M_i = 9\) if node \(i\) corresponds to a functional predictor and \(M_j = s\) if node \(j\) corresponds to a scalar predictor. 
	
	For each functional predictor, we assgin  \(M\) reference eigenvalues (or FPC score) drawn independently from gamma distributions with decreasing means.
	These reference eigenvalues will later be multiplies by randomly  multiplier drawn from correlated multivaraite Normal distribution. These reference eigenvalues are fixed over all samples and all independent repetition of the experiment. We draw it one randomly just to ensure the differentiation between the two functional predictors.
	we independently draw for each functional predictor from . We then sample zero-mean multivariate normal data from the covariance matrix \(\mathbf{R}\). The last \(3s\) components are assigned as scalar predictors, while the first \(2M\) components, scaled by their corresponding eigenvalues, serve as FPC scores. These scores are then expanded into functional data, evaluated over 100 equally spaced points on \([0,1]\) using a Fourier basis.  
	
	Finally, we introduce structured variability by adding a common mean function, defined as a scaled and shifted sine function. To visualize the generated data, Figure \ref{fig:correlation_graph_comparison}  plots functional predictor realizations and heatmaps of sample correlation matrices for both graph structures, illustrating the distinct dependency patterns induced by the precision matrices.  
	
	
	
	\subsection{Simulation shown in february 2025 meeting}\label{simul:meeting_feb}
	To effectively manage the dependencies among functional predictors, scalar covariates, and between functional and scalar predictors, we utilize a Gaussian Markov random field (GMRF) within the framework of Gaussian undirected graphical models. In a GMRF, the off-diagonal elements of the precision matrix capture the conditional correlations between the corresponding components. Given that our setting involves both functional and scalar covariates, we adopt the simulation setup proposed by \cite{kolar_graph_2014}, which focuses on mixed attribute Gaussian graphical models.
	
	We construct a graph consisting of two functional predictor nodes and three vector predictor nodes. Below, we sequentially describe the graph structure and the generation process for each node.
	
	\medskip
	\noindent
	\textit{Functional Predictors.}
	We consider two functional predictors that share the same precision matrix. Denoted as \( \boldsymbol{\Theta} := (\theta_{t \tilde{t}}) \in \mathbb{R}^{p \times p} \), this precision matrix follows an AR(1) structure with a white noise variance of \( 0.1^2 \) and an autoregressive coefficient of \( = 0.95 \), ensuring a smooth functional trajectory.
	
	More formally, each functional predictor is evaluated at \( p \) equally spaced points on \( [0,1] \), with values defined recursively as:
	\begin{equation*}
		X^{(k)}_i(0) = 0, \quad 
		X^{(k)}_i(t)= \rho X^{(k)}_i(t-1) + \varepsilon_{itk}, \quad \varepsilon_{itk} \overset{\text{i.i.d.}}{\sim} \mathcal{N}(0, \sigma^2), \quad t = 1, \dots, p,~i=1, \ldots, n,~k = 1,2.
	\end{equation*}
	In this setting, the precision matrix \( \boldsymbol{\Theta} \) exhibits a tridiagonal Toeplitz structure, where the diagonal entries are given by:
	\begin{equation*}
		\theta_{tt} = 
		\begin{cases}
			\dfrac{1}{\sigma^2(1-0.95^2)}, & t = 1, p, \\
			\dfrac{1+0.95^2}{\sigma^2(1-0.95^2)}, & 2 \leq t \leq p-1.
		\end{cases}
	\end{equation*}
	The off-diagonal entries are:
	\begin{equation*}
		\theta_{t, t+1} = \theta_{t+1, t} = -\dfrac{\rho}{\sigma^2(1-\rho^2)}, \quad 1 \leq t \leq p-1.
	\end{equation*}
	Finally, the functional predictors are smoothed using a B-spline basis with 15 basis functions, as described in Section \ref{section:sub:compute_PLS_component}.
	
	\medskip
	\noindent
	\textit{Vector Predictors.}
	We consider \( s \) vector predictors, each following a \( d \)-dimensional zero-mean Gaussian distribution with a shared precision matrix. This precision matrix, denoted as \( \Gamma := (\gamma_{ij}) \in \mathbb{R}^{d \times d} \), follows a Toeplitz structure with exponentially decaying entries, given by:
	\begin{equation*}  
		\gamma_{ij} := 0.5^{|i-j|}.
	\end{equation*}
	
	\medskip
	\noindent
	\textit{Graph Structure.}
	We arrange five nodes   in a chain structure, where each node follows a sequential order, and the last node connects back to the first, as illustrated in Figure~\ref{fig:graph}. An edge in the graph indicates that the connected nodes remain correlated when the values of all other nodes are fixed. 
	The resulting marginal dependence structure is significantly more complex than the chain structure itself. We designate nodes \(F_1\) and \(F_2\) as functional predictors and nodes \(V_1\), \(V_2\), and \(V_3\) as vector predictors.
\begin{comment}
	\begin{figure}
		\centering
		\includegraphics[width=0.4\linewidth]{image/graph}
		\caption{}
		\label{fig:graph}
	\end{figure}
\end{comment}
	%
	To introduce conditional dependencies between the components, we set the off-diagonal blocks of the precision matrix to be $0.5 \mathbf{1}$ if the corresponding components are connected in the graph, and zero otherwise. Here, $\mathbf{1}$ denotes a matrix of appropriate dimensions where all elements are equal to 1. Overall, we generate a $2p + 3d$-dimensional multivariate Gaussian distribution with mean zero and a precision matrix $\Omega$, structured as follows:
	
	\[
	\Omega =
	\begin{pmatrix}
		\Omega_{F} & 0.5 \mathbf{1} & 0 & 0 & 0.5 \mathbf{1} \\
		0.5 \mathbf{1} & \Omega_{F} & 0.5 \mathbf{1} & 0 & 0 \\
		0 & 0.5 \mathbf{1} & \Omega_{V} & 0.5 \mathbf{1} & 0 \\
		0 & 0 & 0.5 \mathbf{1} & \Omega_{V} & 0.5 \mathbf{1} \\
		0.5 \mathbf{1} & 0 & 0 & 0.5 \mathbf{1} & \Omega_{V} \\
	\end{pmatrix}
	\]
	
	For each observation drawn from this multivariate Gaussian distribution, we process 
	
	The regression coefficients for the first functional predictor, $\beta_{F_1}$, are drawn from a multivariate normal distribution $N(0, 5 \mathbf{I}_p)$, with a fixed random seed. The coefficients for the second functional predictor, $\beta_{F_2}$, are drawn independently from the same distribution and are also smoothed using a B-spline basis with 10 basis functions.
	For the vector covariates, the regression coefficients are sampled from $N(0, I_d)$. After calculating the inner product of the covariates and their corresponding coefficients, independent Gaussian noise from $N(0, 0.1^2)$ is added to the generated responses to simulate measurement noise.
	
	The baseline methods are:
	\begin{itemize}
		\item Penalized functional regression (pfr)~\cite
		{goldsmithPenalizedFunctionalRegression2011}\item Principal component regression (fpcr): run both of PCA for multple functional predictors~\cite{happ_multivariate_2018} and scalar PCA and run OLS on the PC scores.
	\end{itemize}
	% TODO: \usepackage{graphicx} required
	
	We compare our method with these baseline methods using $p=100$. We consider scenarios with $d=1, 2, 3, 4, 5$ and $n = 100, 200, 300, 400$. For each scenario, we use 70\% of the data for training and evaluate prediction performance on the remaining 30\% test set, using the root prediction mean squared error as the evaluation metric. For our method and PCR, the maximum number of components are set as 20. The number of components is chosen by 5-fold cross validation. The results, summarized in Table~\ref{table
	}, demonstrate that our method consistently outperforms the baseline methods across all scenarios.
	\begin{comment}
	\begin{figure}
		\centering
		\includegraphics[width=0.9\linewidth]{Screenshot 2025-02-25 125839.png}
		\caption{Enter Caption}
		\label{fig:enter-label}
	\end{figure}
	\end{comment}
	
	\section{Data Application} 
	
	\medskip
	\noindent
	\textit{Renal study data.}~
	We applied our proposed hybrid functional PLS regression, along with other regression methods, to the Emory renal study data. The study collected data on 226 kidneys (left and right) from 113 subjects, including: (i) baseline renogram curves; (ii) post-furosemide renogram curves; (iii) ordinal ratings of kidney obstruction status (non-obstructed, equivocal, or obstructed) independently assessed by three nuclear medicine experts; (iv) eight kidney-level pharmacokinetic variables derived from radionuclide imaging; and (v) two subject-level variables (age and gender). The subjects had a mean age of 57.8 years (SD = 15.5; range = 18–83), with 54 males (48\%) and 59 females (52\%). The three experts unanimously classified 153 kidneys as non-obstructed, 5 as equivocal, and 40 as obstructed, while 28 kidneys had discrepant ratings.
	
	The two renogram curves, (i) and (ii), were treated as functional predictors and smoothed using a B-spline basis of order 15. The remaining variables, excluding the diagnosis, were treated as scalar predictors. Given the nature of these variables, we assume they are correlated with the renogram curves but not entirely redundant, as they may contain additional useful information. Finally, the diagnoses provided by the three experts were averaged and transformed using a min-max logit transformation. We splitted the data into 70\% of training data and 30\% of testing data, and evalauted the prediction perforamnec by root mean squared error on the test data, normalized by the range of the test data response.
	
	
	
	\subsection{}
			...
			
			% References
			\newpage
			\bibliography{bibliography.bib}
			
			% Tables and Figures at the end
			\newpage
			\section*{Tables and Figures}
			
			% Example Table
			\begin{table}[htbp]
				\centering
				\caption{Example Table Caption.}
				\begin{tabular}{lll}
					\hline
					Col1 & Col2 & Col3 \\
					\hline
					a & b & c \\
					\hline
				\end{tabular}
			\end{table}
			
			% Example Figure
			\begin{figure}[htbp]
				\centering
				\rule{3in}{2in} % placeholder box
				\caption{Example Figure Caption.}
			\end{figure}
			
			% Uncomment to enforce max number check at compile time
			%\ifnum\value{totalfigtab}>6
			%   \errmessage{Too many tables/figures: maximum is 6.}
			%\fi
			
		\end{document}
